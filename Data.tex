\documentclass[]{article}
\usepackage{lmodern}
\usepackage{amssymb,amsmath}
\usepackage{ifxetex,ifluatex}
\usepackage{fixltx2e} % provides \textsubscript
\ifnum 0\ifxetex 1\fi\ifluatex 1\fi=0 % if pdftex
  \usepackage[T1]{fontenc}
  \usepackage[utf8]{inputenc}
\else % if luatex or xelatex
  \ifxetex
    \usepackage{mathspec}
    \usepackage{xltxtra,xunicode}
  \else
    \usepackage{fontspec}
  \fi
  \defaultfontfeatures{Mapping=tex-text,Scale=MatchLowercase}
  \newcommand{\euro}{€}
\fi
% use upquote if available, for straight quotes in verbatim environments
\IfFileExists{upquote.sty}{\usepackage{upquote}}{}
% use microtype if available
\IfFileExists{microtype.sty}{%
\usepackage{microtype}
\UseMicrotypeSet[protrusion]{basicmath} % disable protrusion for tt fonts
}{}
\usepackage[margin=1in]{geometry}
\usepackage{color}
\usepackage{fancyvrb}
\newcommand{\VerbBar}{|}
\newcommand{\VERB}{\Verb[commandchars=\\\{\}]}
\DefineVerbatimEnvironment{Highlighting}{Verbatim}{commandchars=\\\{\}}
% Add ',fontsize=\small' for more characters per line
\usepackage{framed}
\definecolor{shadecolor}{RGB}{248,248,248}
\newenvironment{Shaded}{\begin{snugshade}}{\end{snugshade}}
\newcommand{\KeywordTok}[1]{\textcolor[rgb]{0.13,0.29,0.53}{\textbf{{#1}}}}
\newcommand{\DataTypeTok}[1]{\textcolor[rgb]{0.13,0.29,0.53}{{#1}}}
\newcommand{\DecValTok}[1]{\textcolor[rgb]{0.00,0.00,0.81}{{#1}}}
\newcommand{\BaseNTok}[1]{\textcolor[rgb]{0.00,0.00,0.81}{{#1}}}
\newcommand{\FloatTok}[1]{\textcolor[rgb]{0.00,0.00,0.81}{{#1}}}
\newcommand{\CharTok}[1]{\textcolor[rgb]{0.31,0.60,0.02}{{#1}}}
\newcommand{\StringTok}[1]{\textcolor[rgb]{0.31,0.60,0.02}{{#1}}}
\newcommand{\CommentTok}[1]{\textcolor[rgb]{0.56,0.35,0.01}{\textit{{#1}}}}
\newcommand{\OtherTok}[1]{\textcolor[rgb]{0.56,0.35,0.01}{{#1}}}
\newcommand{\AlertTok}[1]{\textcolor[rgb]{0.94,0.16,0.16}{{#1}}}
\newcommand{\FunctionTok}[1]{\textcolor[rgb]{0.00,0.00,0.00}{{#1}}}
\newcommand{\RegionMarkerTok}[1]{{#1}}
\newcommand{\ErrorTok}[1]{\textbf{{#1}}}
\newcommand{\NormalTok}[1]{{#1}}
\usepackage{longtable,booktabs}
\ifxetex
  \usepackage[setpagesize=false, % page size defined by xetex
              unicode=false, % unicode breaks when used with xetex
              xetex]{hyperref}
\else
  \usepackage[unicode=true]{hyperref}
\fi
\hypersetup{breaklinks=true,
            bookmarks=true,
            pdfauthor={jcb},
            pdftitle={Superviser un étudiant dans son travail de recherche},
            colorlinks=true,
            citecolor=blue,
            urlcolor=blue,
            linkcolor=magenta,
            pdfborder={0 0 0}}
\urlstyle{same}  % don't use monospace font for urls
\setlength{\parindent}{0pt}
\setlength{\parskip}{6pt plus 2pt minus 1pt}
\setlength{\emergencystretch}{3em}  % prevent overfull lines
\setcounter{secnumdepth}{5}

%%% Use protect on footnotes to avoid problems with footnotes in titles
\let\rmarkdownfootnote\footnote%
\def\footnote{\protect\rmarkdownfootnote}

%%% Change title format to be more compact
\usepackage{titling}
\setlength{\droptitle}{-2em}
  \title{Superviser un étudiant dans son travail de recherche}
  \pretitle{\vspace{\droptitle}\centering\huge}
  \posttitle{\par}
  \author{jcb}
  \preauthor{\centering\large\emph}
  \postauthor{\par}
  \predate{\centering\large\emph}
  \postdate{\par}
  \date{10 février 2015}




\begin{document}

\maketitle


{
\hypersetup{linkcolor=black}
\setcounter{tocdepth}{2}
\tableofcontents
}
\section{Introduction}\label{introduction}

\href{http://simplystatistics.org/2013/12/16/a-summary-of-the-evidence-that-most-published-research-is-false/}{A
summary of the evidence that most published research is false}

\section{Organiser son travail}\label{organiser-son-travail}

Stafford Noble W.
\href{http://journals.plos.org/ploscompbiol/article?id=10.1371/journal.pcbi.1000424}{A
quick Guide to Organizing Computational Biology Projects}. PLOS Comp.
Biol. (2009) \textbf{5}(7). Accédé le 10/2/2015.

Un
\href{https://github.com/chendaniely/computational-project-cookie-cutter}{script}
pour implémenter automatiquement la structure proposée dans l'article de
Stafford Noble.

Structuring a Data Analysis using R (Part 1 of 2) --
\href{http://datatechblog.com/2013/12/data-analysis-using-r-gathering-organizing-munging/}{Gathering,
Organizing, Exploring and Munging Data} Structuring a Data Analysis
using R (Part 2 of 2) --
\href{http://datatechblog.com/2014/01/structuring-data-analysis-using-r-part-2-2-analyzing-modeling-write/}{Analyzing,
Modeling, and the Write-up}

\section{Saisir les données}\label{saisir-les-donnees}

Faut-il utiliser un tableur ou un logiciel de statistiques pour saisir
les données ?
\href{http://http://www.quantumforest.com/2013/12/excel-fanaticism-and-r/}{Luis
A. Apiolaza}

\subsection{Anatomie d'un tableur}\label{anatomie-dun-tableur}

\href{http://www.ilri.org/rmg/RMGcd/DataAnalysisAgroforestry/Materials/Other\%20documents/GenstatManual.pdf}{Manuel
Genstat}

\section{Divers}\label{divers}

\subsection{Organiser le versionning}\label{organiser-le-versionning}

\href{http://git-scm.com/book/fr/}{Git} et Github.

Git-les bases pourbbien gérer les versions de votre projet. Linux
Pratique (2013) n°83 59-63.

Représenter les données: notion UTF8

Une des premières tâches dans un projet de recherche est de savoir
comment recueillir et stocker ses données de manière intelligible et
efficiente: - intelligible: comprendre de ce que l'on a fait plusieurs
mois plus tard - efficiente: 80\% du temps nécessaire au traitement
statistique des données est consacré à nettoyer les données (\#ref)

Règle 1: le support de saisie doit servir uniquement à cette tâche et
rien d'autre.

Les données nettoyées: ce sont typiquement celles que l'on trouve dans
les exemples des ouvrages de référence. Elles sont ``parfaites'' en ce
sens qu``elles peuvent être importée dans un logiciel de statistique et
traitées immédiatement sans que le progrmme de génère une erreur.
L'objectf est que vos donnnées soient''parfaites``. Hélas dans la vraie
vie rien n'est parfait et il est rare q'un logiciels accepte sans
broncher vos données.

\subsection{Les données manquantes}\label{les-donnees-manquantes}

Des données peuvent mnquer pour de nombreuses raisons. La plupart des
logiciels acceptent le symbole NA (pour not avalaible) pour désigner une
valeur manquante. Pour obtenir un vecteur de valeurs non manquantes:
x{[}!is.na(x){]}.

Référence:

\href{http://www.nap.edu/catalog/18267/sharing-clinical-research-data-workshop-summary}{Sharing
Clinical Research Data: Workshop Summary}. Accédé le 7/2/2015

\section{Data}\label{data}

Hadley Wickham \href{http://www.jstatsoft.org/v59/i10/paper}{Tidy Data}

Une version alégée de
\href{https://ramnathv.github.io/pycon2014-r/explore/tidy.html}{cet
article} figure ci-après:

\subsection{Données ordonnées (Tidy
data)}\label{donnees-ordonnees-tidy-data}

Ce sont les tableaux de données que l'on touve dans les ouvrages de
statistiques. Ils ont prêts à l'emploi, sans ereurs, sans données
manquantes. Soumises à un logiciel de statistiques, elle produiront
instantannément un résultat sans un arrêt intempestif du programme
assorti de message d'erreur plus ou moins abscons.

Les données brutes (Raw data) dans le monde réel sont souvent
désordonnés et mal formatées (Messy data). En outre, il peut manquer de
détails appropriés de l'étude. La correction des données natives peut
être un exercice périlleux puisque les données brutes originales risque
d'être écrasées et il n'y aurait aucun moyen de vérifier ce processus ou
de récupérer des erreurs commises pendant cette phase. Une bonne
pratique serait de conserver les données d'origine, et d'utiliser un
script programmatique pour nettoyer, corriger les erreurs et sauvegarder
cet ensemble de données nettoyées pour une analyse ultérieure.

Question: combien de variable dans le tableau suivant:

\begin{longtable}[c]{@{}lll@{}}
\toprule\addlinespace
lésions & hommes & femmes
\\\addlinespace
\midrule\endhead
oui & 4 & 1
\\\addlinespace
non & 2 & 5
\\\addlinespace
\bottomrule
\end{longtable}

La façon dont le tableau est présenté, il semble que il ya seulement
deux variables. La bonne réponse est trois: lésions?, sexe, nombre.

\begin{longtable}[c]{@{}llll@{}}
\toprule\addlinespace
sujet & lésion & sexe & nombre
\\\addlinespace
\midrule\endhead
1 & oui & H & 4
\\\addlinespace
2 & oui & F & 1
\\\addlinespace
3 & non & H & 2
\\\addlinespace
4 & non & F & 5
\\\addlinespace
\bottomrule
\end{longtable}

Comment définir des données bien ordonnées ? Ce sont des données
répondant aux 3 caractéristiques suivantes:

\begin{itemize}
\itemsep1pt\parskip0pt\parsep0pt
\item
  les observations (sujets) sont en lignes
\item
  les variables sont en colonnes
\item
  toutes les données sont dans le même tableau
\end{itemize}

Ce format est également appelé large (wide) et s'oppose au format long
(long). La plupart des logiciels ont des instructions permettant de
passer d'un format à l'autre (ex. format pivot dans excel).

Les données bien ordonnées facilitent l'analyse statistiques des
données.

Les erreurs les plus fréquentes:

Répartition des revenus par Religion

Notre premier ensemble de données est basée sur une enquête réalisée par
\href{http://www.pewforum.org/2009/01/30/income-distribution-within-us-religious-groups/}{Pew
Research} qui examine la relation entre le revenu et l'appartenance
religieuse.

\begin{verbatim}
religion  <$10k $10-20k $20-30k $30-40k $40-50k $50-75k $75-100k    $100-150k   >150k   Don't know/refused
1   Agnostic    27  34  60  81  76  137 122 109 84  96
2   Atheist 12  27  37  52  35  70  73  59  74  76
3   Buddhist    27  21  30  34  33  58  62  39  53  54
4   Catholic    418 617 732 670 638 1116    949 792 633 1489
5   Don’t know/refused  15  14  15  11  10  35  21  17  18  116
6   Evangelical Prot    575 869 1064    982 881 1486    949 723 414 1529
7   Hindu   1   9   7   9   11  34  47  48  54  37
8   Historically Black Prot 228 244 236 238 197 223 131 81  78  339
9   Jehovah's Witness   20  27  24  24  21  30  15  11  6   37
10  Jewish  19  19  25  25  30  95  69  87  151 162
11  Mainline Prot   289 495 619 655 651 1107    939 753 634 1328
12  Mormon  29  40  48  51  56  112 85  49  42  69
13  Muslim  6   7   9   10  9   23  16  8   6   22
14  Orthodox    13  17  23  32  32  47  38  42  46  73
15  Other Christian 9   7   11  13  13  14  18  14  12  18
16  Other Faiths    20  33  40  46  49  63  46  40  41  71
17  Other World Religions   5   2   3   4   2   7   3   4   4   8
18  Unaffiliated    217 299 374 365 341 528 407 321 258 597
\end{verbatim}

Tuberculose

\begin{verbatim}
##   iso2 year new_sp new_sp_m04 new_sp_m514 new_sp_m014 new_sp_m1524
## 1   AD 1989     NA         NA          NA          NA           NA
## 2   AD 1990     NA         NA          NA          NA           NA
## 3   AD 1991     NA         NA          NA          NA           NA
## 4   AD 1992     NA         NA          NA          NA           NA
## 5   AD 1993     15         NA          NA          NA           NA
## 6   AD 1994     24         NA          NA          NA           NA
##   new_sp_m2534 new_sp_m3544 new_sp_m4554 new_sp_m5564 new_sp_m65 new_sp_mu
## 1           NA           NA           NA           NA         NA        NA
## 2           NA           NA           NA           NA         NA        NA
## 3           NA           NA           NA           NA         NA        NA
## 4           NA           NA           NA           NA         NA        NA
## 5           NA           NA           NA           NA         NA        NA
## 6           NA           NA           NA           NA         NA        NA
##   new_sp_f04 new_sp_f514 new_sp_f014 new_sp_f1524 new_sp_f2534
## 1         NA          NA          NA           NA           NA
## 2         NA          NA          NA           NA           NA
## 3         NA          NA          NA           NA           NA
## 4         NA          NA          NA           NA           NA
## 5         NA          NA          NA           NA           NA
## 6         NA          NA          NA           NA           NA
##   new_sp_f3544 new_sp_f4554 new_sp_f5564 new_sp_f65 new_sp_fu
## 1           NA           NA           NA         NA        NA
## 2           NA           NA           NA         NA        NA
## 3           NA           NA           NA         NA        NA
## 4           NA           NA           NA         NA        NA
## 5           NA           NA           NA         NA        NA
## 6           NA           NA           NA         NA        NA
\end{verbatim}

Sauf pour ISO2 et l'année, le reste des en-têtes de colonnes sont en
fait des valeurs d'une variable qui se cache, en fait la combinaison de
deux variables, le sexe et l'âge.

Le temps

\begin{verbatim}
##            id year month element d1  d2  d3 d4  d5 d6 d7 d8 d9 d10 d11 d12
## 1 MX000017004 2010     1    TMAX NA  NA  NA NA  NA NA NA NA NA  NA  NA  NA
## 2 MX000017004 2010     1    TMIN NA  NA  NA NA  NA NA NA NA NA  NA  NA  NA
## 3 MX000017004 2010     2    TMAX NA 273 241 NA  NA NA NA NA NA  NA 297  NA
## 4 MX000017004 2010     2    TMIN NA 144 144 NA  NA NA NA NA NA  NA 134  NA
## 5 MX000017004 2010     3    TMAX NA  NA  NA NA 321 NA NA NA NA 345  NA  NA
## 6 MX000017004 2010     3    TMIN NA  NA  NA NA 142 NA NA NA NA 168  NA  NA
##   d13 d14 d15 d16 d17 d18 d19 d20 d21 d22 d23 d24 d25 d26 d27 d28 d29 d30
## 1  NA  NA  NA  NA  NA  NA  NA  NA  NA  NA  NA  NA  NA  NA  NA  NA  NA 278
## 2  NA  NA  NA  NA  NA  NA  NA  NA  NA  NA  NA  NA  NA  NA  NA  NA  NA 145
## 3  NA  NA  NA  NA  NA  NA  NA  NA  NA  NA 299  NA  NA  NA  NA  NA  NA  NA
## 4  NA  NA  NA  NA  NA  NA  NA  NA  NA  NA 107  NA  NA  NA  NA  NA  NA  NA
## 5  NA  NA  NA 311  NA  NA  NA  NA  NA  NA  NA  NA  NA  NA  NA  NA  NA  NA
## 6  NA  NA  NA 176  NA  NA  NA  NA  NA  NA  NA  NA  NA  NA  NA  NA  NA  NA
##   d31
## 1  NA
## 2  NA
## 3  NA
## 4  NA
## 5  NA
## 6  NA
\end{verbatim}

Cet ensemble de données semble avoir deux problèmes. Premièrement, il a
variables dans les lignes de l'élément de colonne. Deuxièmement, il a un
d variable dans l'en-tête de colonne répartis sur plusieurs colonnes.

Les erreurs les plus fréquentes:

\begin{itemize}
\itemsep1pt\parskip0pt\parsep0pt
\item
  les entêtes de colonnes sont des valeurs, pas les noms de variables
\item
  Plusieurs variables sont stockées dans une colonne
\item
  Les variables sont stockées dans les lignes et les colonnes
\item
  Plusieurs types de unité expérimentale stockés dans la même table
\item
  Un type d'appareil d'essai stockées dans plusieurs tables
\end{itemize}

\href{https://ramnathv.github.io/pycon2014-r/explore/reshape.html}{suite}

Rmodeler (reshaping)

Une façon de données est bien rangé pour remodeler pour qu'il adhère aux
trois règles de données bien rangé. Bien que la base R a plusieurs
fonctions visant à remodeler les données, nous allons utiliser le
package reshape2 par Hadley Wickham, car il fournit un ensemble simple
et cohérente des fonctions pour remodeler données. Notions de base

En termes simples, le remodelage des données est comme faire un tableau
croisé dynamique (pivot) dans Excel, où vous mélangez colonnes, des
lignes et des valeurs. Commençons par nous ranger l'ensemble de données
de pew.

We can tidy this data using the melt function in the reshape2 package.

\begin{Shaded}
\begin{Highlighting}[]
\KeywordTok{library}\NormalTok{(reshape2)}
\NormalTok{pew_tidy <-}\StringTok{ }\KeywordTok{melt}\NormalTok{(}
  \DataTypeTok{data =} \NormalTok{pew,}
  \DataTypeTok{id =} \StringTok{"religion"}\NormalTok{,}
  \DataTypeTok{variable.name =} \StringTok{"income"}\NormalTok{,}
  \DataTypeTok{value.name =} \StringTok{"frequency"}
\NormalTok{)}
\end{Highlighting}
\end{Shaded}

\subsection{\href{http://cran.r-project.org/web/packages/tidyr/vignettes/tidy-data.html}{Autre
référence}}\label{autre-reference}

Data tidying

It is often said that 80\% of data analysis is spent on the cleaning and
preparing data. And it's not just a first step, but it must be repeated
many over the course of analysis as new problems come to light or new
data is collected. To get a handle on the problem, this paper focuses on
a small, but important, aspect of data cleaning that I call data
tidying: structuring datasets to facilitate analysis.

The principles of tidy data provide a standard way to organise data
values within a dataset. A standard makes initial data cleaning easier
because you don't need to start from scratch and reinvent the wheel
every time. The tidy data standard has been designed to facilitate
initial exploration and analysis of the data, and to simplify the
development of data analysis tools that work well together. Current
tools often require translation. You have to spend time munging the
output from one tool so you can input it into another. Tidy datasets and
tidy tools work hand in hand to make data analysis easier, allowing you
to focus on the interesting domain problem, not on the uninteresting
logistics of data. Defining tidy data \{\#sec:defining\}

\begin{verbatim}
Happy families are all alike; every unhappy family is unhappy in its own way — Leo Tolstoy
\end{verbatim}

Like families, tidy datasets are all alike but every messy dataset is
messy in its own way. Tidy datasets provide a standardized way to link
the structure of a dataset (its physical layout) with its semantics (its
meaning). In this section, I'll provide some standard vocabulary for
describing the structure and semantics of a dataset, and then use those
definitions to define tidy data. Data structure

Most statistical datasets are data frames made up of rows and columns.
The columns are almost always labeled and the rows are sometimes
labeled. The following code provides some data about an imaginary
experiment in a format commonly seen in the wild. The table has two
columns and three rows, and both rows and columns are labeled.

preg \textless{}- read.csv(``preg.csv'', stringsAsFactors = FALSE) preg
\#\textgreater{} name treatmenta treatmentb \#\textgreater{} 1 John
Smith NA 18 \#\textgreater{} 2 Jane Doe 4 1 \#\textgreater{} 3 Mary
Johnson 6 7

There are many ways to structure the same underlying data. The following
table shows the same data as above, but the rows and columns have been
transposed.

read.csv(``preg2.csv'', stringsAsFactors = FALSE) \#\textgreater{}
treatment John.Smith Jane.Doe Mary.Johnson \#\textgreater{} 1 a NA 4 6
\#\textgreater{} 2 b 18 1 7

The data is the same, but the layout is different. Our vocabulary of
rows and columns is simply not rich enough to describe why the two
tables represent the same data. In addition to appearance, we need a way
to describe the underlying semantics, or meaning, of the values
displayed in table. Data semantics

A dataset is a collection of values, usually either numbers (if
quantitative) or strings (if qualitative). Values are organised in two
ways. Every value belongs to a variable and an observation. A variable
contains all values that measure the same underlying attribute (like
height, temperature, duration) across units. An observation contains all
values measured on the same unit (like a person, or a day, or a race)
across attributes.

A tidy version of the pregnancy data looks like this: (you'll learn how
the functions work a little later)

library(tidyr) library(dplyr) preg2 \textless{}- preg \%\textgreater{}\%
gather(treatment, n, treatmenta:treatmentb) \%\textgreater{}\%
mutate(treatment = gsub(``treatment'', ``'', treatment))
\%\textgreater{}\% arrange(name, treatment) preg2 \#\textgreater{} name
treatment n \#\textgreater{} 1 Jane Doe a 4 \#\textgreater{} 2 Jane Doe
b 1 \#\textgreater{} 3 John Smith a NA \#\textgreater{} 4 John Smith b
18 \#\textgreater{} 5 Mary Johnson a 6 \#\textgreater{} 6 Mary Johnson b
7

This makes the values, variables and observations more clear. The
dataset contains 18 values representing three variables and six
observations. The variables are:

\begin{verbatim}
name, with three possible values (John, Mary, and Jane).

treatment, with two possible values (a and b).

n, with five or six values depending on how you think of the missing value (1, 4, 6, 7, 18, NA)
\end{verbatim}

The experimental design tells us more about the structure of the
observations. In this experiment, every combination of of person and
treatment was measured, a completely crossed design. The experimental
design also determines whether or not missing values can be safely
dropped. In this experiment, the missing value represents an observation
that should have been made, but wasn't, so it's important to keep it.
Structural missing values, which represent measurements that can't be
made (e.g., the count of pregnant males) can be safely removed.

For a given dataset, it's usually easy to figure out what are
observations and what are variables, but it is surprisingly difficult to
precisely define variables and observations in general. For example, if
the columns in the pregnancy data were height and weight we would have
been happy to call them variables. If the columns were height and width,
it would be less clear cut, as we might think of height and width as
values of a dimension variable. If the columns were home phone and work
phone, we could treat these as two variables, but in a fraud detection
environment we might want variables phone number and number type because
the use of one phone number for multiple people might suggest fraud. A
general rule of thumb is that it is easier to describe functional
relationships between variables (e.g., z is a linear combination of x
and y, density is the ratio of weight to volume) than between rows, and
it is easier to make comparisons between groups of observations (e.g.,
average of group a vs.~average of group b) than between groups of
columns.

In a given analysis, there may be multiple levels of observation. For
example, in a trial of new allergy medication we might have three
observational types: demographic data collected from each person (age,
sex, race), medical data collected from each person on each day (number
of sneezes, redness of eyes), and meteorological data collected on each
day (temperature, pollen count).

Variables may change over the course of analysis. Often the variables in
the raw data are very fine grained, and may add extra modelling
complexity for little explanatory gain. For example, many surveys ask
variations on the same question to better get at an underlying trait. In
early stages of analysis, variables correspond to questions. In later
stages, you change focus to traits, computed by averaging together
multiple questions. This considerably simplifies analysis because you
don't need a hierarchical model, and you can often pretend that the data
is continuous, not discrete. Tidy data

Tidy data is a standard way of mapping the meaning of a dataset to its
structure. A dataset is messy or tidy depending on how rows, columns and
tables are matched up with observations, variables and types. In tidy
data:

\begin{verbatim}
Each variable forms a column.

Each observation forms a row.

Each type of observational unit forms a table.
\end{verbatim}

This is Codd's 3rd normal form, but with the constraints framed in
statistical language, and the focus put on a single dataset rather than
the many connected datasets common in relational databases. Messy data
is any other other arrangement of the data.

Tidy data makes it easy for an analyst or a computer to extract needed
variables because it provides a standard way of structuring a dataset.
Compare the different versions of the pregnancy data: in the messy
version you need to use different strategies to extract different
variables. This slows analysis and invites errors. If you consider how
many data analysis operations involve all of the values in a variable
(every aggregation function), you can see how important it is to extract
these values in a simple, standard way. Tidy data is particularly well
suited for vectorised programming languages like R, because the layout
ensures that values of different variables from the same observation are
always paired.

While the order of variables and observations does not affect analysis,
a good ordering makes it easier to scan the raw values. One way of
organising variables is by their role in the analysis: are values fixed
by the design of the data collection, or are they measured during the
course of the experiment? Fixed variables describe the experimental
design and are known in advance. Computer scientists often call fixed
variables dimensions, and statisticians usually denote them with
subscripts on random variables. Measured variables are what we actually
measure in the study. Fixed variables should come first, followed by
measured variables, each ordered so that related variables are
contiguous. Rows can then be ordered by the first variable, breaking
ties with the second and subsequent (fixed) variables. This is the
convention adopted by all tabular displays in this paper. Tidying messy
datasets \{\#sec:tidying\}

Real datasets can, and often do, violate the three precepts of tidy data
in almost every way imaginable. While occasionally you do get a dataset
that you can start analysing immediately, this is the exception, not the
rule. This section describes the five most common problems with messy
datasets, along with their remedies:

\begin{verbatim}
Column headers are values, not variable names.

Multiple variables are stored in one column.

Variables are stored in both rows and columns.

Multiple types of observational units are stored in the same table.

A single observational unit is stored in multiple tables.
\end{verbatim}

Surprisingly, most messy datasets, including types of messiness not
explicitly described above, can be tidied with a small set of tools:
gathering, separating and spreading. The following sections illustrate
each problem with a real dataset that I have encountered, and show how
to tidy them. Column headers are values, not variable names

A common type of messy dataset is tabular data designed for
presentation, where variables form both the rows and columns, and column
headers are values, not variable names. While I would call this
arrangement messy, in some cases it can be extremely useful. It provides
efficient storage for completely crossed designs, and it can lead to
extremely efficient computation if desired operations can be expressed
as matrix operations.

The following code shows a subset of a typical dataset of this form.
This dataset explores the relationship between income and religion in
the US. It comes from a report1 produced by the Pew Research Center, an
American think-tank that collects data on attitudes to topics ranging
from religion to the internet, and produces many reports that contain
datasets in this format.

pew \textless{}- tbl\_df(read.csv(``pew.csv'', stringsAsFactors = FALSE,
check.names = FALSE)) pew \#\textgreater{} Source: local data frame
{[}18 x 11{]} \#\textgreater{} \#\textgreater{} religion
\textless{}\$10k \$10-20k \$20-30k \$30-40k \$40-50k \$50-75k
\#\textgreater{} 1 Agnostic 27 34 60 81 76 137 \#\textgreater{} 2
Atheist 12 27 37 52 35 70 \#\textgreater{} 3 Buddhist 27 21 30 34 33 58
\#\textgreater{} 4 Catholic 418 617 732 670 638 1116 \#\textgreater{} 5
Don't know/refused 15 14 15 11 10 35 \#\textgreater{} 6 Evangelical Prot
575 869 1064 982 881 1486 \#\textgreater{} 7 Hindu 1 9 7 9 11 34
\#\textgreater{} 8 Historically Black Prot 228 244 236 238 197 223
\#\textgreater{} 9 Jehovah's Witness 20 27 24 24 21 30 \#\textgreater{}
10 Jewish 19 19 25 25 30 95 \#\textgreater{} .. \ldots{} \ldots{}
\ldots{} \ldots{} \ldots{} \ldots{} \ldots{} \#\textgreater{} Variables
not shown: \$75-100k (int), \$100-150k (int), \textgreater{}150k (int),
Don't \#\textgreater{} know/refused (int)

This dataset has three variables, religion, income and frequency. To
tidy it, we need to gather the non-variable columns into a two-column
key-value pair. This action is often described as making a wide dataset
long (or tall), but I'll avoid those terms because they're imprecise.

When gathering variables, we need to provide the name of the new
key-value columns to create. The first argument, is the name of the key
column, which is the name of the variable defined by the values of the
column headings. In this case, it's income. The second argument is the
name of the value column, frequency. The third argument defines the
columns to gather, here, every column except religion.

pew \%\textgreater{}\% gather(income, frequency, -religion)
\#\textgreater{} Source: local data frame {[}180 x 3{]} \#\textgreater{}
\#\textgreater{} religion income frequency \#\textgreater{} 1 Agnostic
\textless{}\$10k 27 \#\textgreater{} 2 Atheist \textless{}\$10k 12
\#\textgreater{} 3 Buddhist \textless{}\$10k 27 \#\textgreater{} 4
Catholic \textless{}\$10k 418 \#\textgreater{} 5 Don't know/refused
\textless{}\$10k 15 \#\textgreater{} 6 Evangelical Prot \textless{}\$10k
575 \#\textgreater{} 7 Hindu \textless{}\$10k 1 \#\textgreater{} 8
Historically Black Prot \textless{}\$10k 228 \#\textgreater{} 9
Jehovah's Witness \textless{}\$10k 20 \#\textgreater{} 10 Jewish
\textless{}\$10k 19 \#\textgreater{} .. \ldots{} \ldots{} \ldots{}

This form is tidy because each column represents a variable and each row
represents an observation, in this case a demographic unit corresponding
to a combination of religion and income.

This format is also used to record regularly spaced observations over
time. For example, the Billboard dataset shown below records the date a
song first entered the billboard top 100. It has variables for artist,
track, date.entered, rank and week. The rank in each week after it
enters the top 100 is recorded in 75 columns, wk1 to wk75. This form of
storage is not tidy, but it is useful for data entry. It reduces
duplication since otherwise each song in each week would need its own
row, and song metadata like title and artist would need to be repeated.
This will be discussed in more depth in (multiple
types){[}\#multiple-types{]}.

billboard \textless{}- tbl\_df(read.csv(``billboard.csv'',
stringsAsFactors = FALSE)) billboard \#\textgreater{} Source: local data
frame {[}317 x 81{]} \#\textgreater{} \#\textgreater{} year artist track
time date.entered wk1 wk2 \#\textgreater{} 1 2000 2 Pac Baby Don't Cry
(Keep\ldots{} 4:22 2000-02-26 87 82 \#\textgreater{} 2 2000 2Ge+her The
Hardest Part Of \ldots{} 3:15 2000-09-02 91 87 \#\textgreater{} 3 2000 3
Doors Down Kryptonite 3:53 2000-04-08 81 70 \#\textgreater{} 4 2000 3
Doors Down Loser 4:24 2000-10-21 76 76 \#\textgreater{} 5 2000 504 Boyz
Wobble Wobble 3:35 2000-04-15 57 34 \#\textgreater{} 6 2000 98\^{}0 Give
Me Just One Nig\ldots{} 3:24 2000-08-19 51 39 \#\textgreater{} 7 2000
A*Teens Dancing Queen 3:44 2000-07-08 97 97 \#\textgreater{} 8 2000
Aaliyah I Don't Wanna 4:15 2000-01-29 84 62 \#\textgreater{} 9 2000
Aaliyah Try Again 4:03 2000-03-18 59 53 \#\textgreater{} 10 2000 Adams,
Yolanda Open My Heart 5:30 2000-08-26 76 76 \#\textgreater{} .. \ldots{}
\ldots{} \ldots{} \ldots{} \ldots{} \ldots{} \ldots{} \#\textgreater{}
Variables not shown: wk3 (int), wk4 (int), wk5 (int), wk6 (int), wk7
\#\textgreater{} (int), wk8 (int), wk9 (int), wk10 (int), wk11 (int),
wk12 (int), wk13 \#\textgreater{} (int), wk14 (int), wk15 (int), wk16
(int), wk17 (int), wk18 (int), wk19 \#\textgreater{} (int), wk20 (int),
wk21 (int), wk22 (int), wk23 (int), wk24 (int), wk25 \#\textgreater{}
(int), wk26 (int), wk27 (int), wk28 (int), wk29 (int), wk30 (int), wk31
\#\textgreater{} (int), wk32 (int), wk33 (int), wk34 (int), wk35 (int),
wk36 (int), wk37 \#\textgreater{} (int), wk38 (int), wk39 (int), wk40
(int), wk41 (int), wk42 (int), wk43 \#\textgreater{} (int), wk44 (int),
wk45 (int), wk46 (int), wk47 (int), wk48 (int), wk49 \#\textgreater{}
(int), wk50 (int), wk51 (int), wk52 (int), wk53 (int), wk54 (int), wk55
\#\textgreater{} (int), wk56 (int), wk57 (int), wk58 (int), wk59 (int),
wk60 (int), wk61 \#\textgreater{} (int), wk62 (int), wk63 (int), wk64
(int), wk65 (int), wk66 (lgl), wk67 \#\textgreater{} (lgl), wk68 (lgl),
wk69 (lgl), wk70 (lgl), wk71 (lgl), wk72 (lgl), wk73 \#\textgreater{}
(lgl), wk74 (lgl), wk75 (lgl), wk76 (lgl)

To tidy this dataset, we first gather together all the wk columns. The
column names give the week and the values are the ranks:

billboard2 \textless{}- billboard \%\textgreater{}\% gather(week, rank,
wk1:wk76, na.rm = TRUE) billboard2 \#\textgreater{} Source: local data
frame {[}5,307 x 7{]} \#\textgreater{} \#\textgreater{} year artist
track time date.entered week rank \#\textgreater{} 1 2000 2 Pac Baby
Don't Cry (Keep\ldots{} 4:22 2000-02-26 wk1 87 \#\textgreater{} 2 2000
2Ge+her The Hardest Part Of \ldots{} 3:15 2000-09-02 wk1 91
\#\textgreater{} 3 2000 3 Doors Down Kryptonite 3:53 2000-04-08 wk1 81
\#\textgreater{} 4 2000 3 Doors Down Loser 4:24 2000-10-21 wk1 76
\#\textgreater{} 5 2000 504 Boyz Wobble Wobble 3:35 2000-04-15 wk1 57
\#\textgreater{} 6 2000 98\^{}0 Give Me Just One Nig\ldots{} 3:24
2000-08-19 wk1 51 \#\textgreater{} 7 2000 A*Teens Dancing Queen 3:44
2000-07-08 wk1 97 \#\textgreater{} 8 2000 Aaliyah I Don't Wanna 4:15
2000-01-29 wk1 84 \#\textgreater{} 9 2000 Aaliyah Try Again 4:03
2000-03-18 wk1 59 \#\textgreater{} 10 2000 Adams, Yolanda Open My Heart
5:30 2000-08-26 wk1 76 \#\textgreater{} .. \ldots{} \ldots{} \ldots{}
\ldots{} \ldots{} \ldots{} \ldots{}

Here we use na.rm to drop any missing values from the gather columns. In
this data, missing values represent weeks that the song wasn't in the
charts, so can be safely dropped.

In this case it's also nice to do a little cleaning, converting the week
variable to a number, and figuring out the date corresponding to each
week on the charts:

billboard3 \textless{}- billboard2 \%\textgreater{}\% mutate( week =
extract\_numeric(week), date = as.Date(date.entered) + 7 * (week - 1))
\%\textgreater{}\% select(-date.entered) billboard3 \#\textgreater{}
Source: local data frame {[}5,307 x 7{]} \#\textgreater{}
\#\textgreater{} year artist track time week rank date \#\textgreater{}
1 2000 2 Pac Baby Don't Cry (Keep\ldots{} 4:22 1 87 2000-02-26
\#\textgreater{} 2 2000 2Ge+her The Hardest Part Of \ldots{} 3:15 1 91
2000-09-02 \#\textgreater{} 3 2000 3 Doors Down Kryptonite 3:53 1 81
2000-04-08 \#\textgreater{} 4 2000 3 Doors Down Loser 4:24 1 76
2000-10-21 \#\textgreater{} 5 2000 504 Boyz Wobble Wobble 3:35 1 57
2000-04-15 \#\textgreater{} 6 2000 98\^{}0 Give Me Just One Nig\ldots{}
3:24 1 51 2000-08-19 \#\textgreater{} 7 2000 A*Teens Dancing Queen 3:44
1 97 2000-07-08 \#\textgreater{} 8 2000 Aaliyah I Don't Wanna 4:15 1 84
2000-01-29 \#\textgreater{} 9 2000 Aaliyah Try Again 4:03 1 59
2000-03-18 \#\textgreater{} 10 2000 Adams, Yolanda Open My Heart 5:30 1
76 2000-08-26 \#\textgreater{} .. \ldots{} \ldots{} \ldots{} \ldots{}
\ldots{} \ldots{} \ldots{}

Finally, it's always a good idea to sort the data. We could do it by
artist, track and week:

billboard3 \%\textgreater{}\% arrange(artist, track, week)
\#\textgreater{} Source: local data frame {[}5,307 x 7{]}
\#\textgreater{} \#\textgreater{} year artist track time week rank date
\#\textgreater{} 1 2000 2 Pac Baby Don't Cry (Keep\ldots{} 4:22 1 87
2000-02-26 \#\textgreater{} 2 2000 2 Pac Baby Don't Cry (Keep\ldots{}
4:22 2 82 2000-03-04 \#\textgreater{} 3 2000 2 Pac Baby Don't Cry
(Keep\ldots{} 4:22 3 72 2000-03-11 \#\textgreater{} 4 2000 2 Pac Baby
Don't Cry (Keep\ldots{} 4:22 4 77 2000-03-18 \#\textgreater{} 5 2000 2
Pac Baby Don't Cry (Keep\ldots{} 4:22 5 87 2000-03-25 \#\textgreater{} 6
2000 2 Pac Baby Don't Cry (Keep\ldots{} 4:22 6 94 2000-04-01
\#\textgreater{} 7 2000 2 Pac Baby Don't Cry (Keep\ldots{} 4:22 7 99
2000-04-08 \#\textgreater{} 8 2000 2Ge+her The Hardest Part Of \ldots{}
3:15 1 91 2000-09-02 \#\textgreater{} 9 2000 2Ge+her The Hardest Part Of
\ldots{} 3:15 2 87 2000-09-09 \#\textgreater{} 10 2000 2Ge+her The
Hardest Part Of \ldots{} 3:15 3 92 2000-09-16 \#\textgreater{} ..
\ldots{} \ldots{} \ldots{} \ldots{} \ldots{} \ldots{} \ldots{}

Or by date and rank:

billboard3 \%\textgreater{}\% arrange(date, rank) \#\textgreater{}
Source: local data frame {[}5,307 x 7{]} \#\textgreater{}
\#\textgreater{} year artist track time week rank date \#\textgreater{}
1 2000 Lonestar Amazed 4:25 1 81 1999-06-05 \#\textgreater{} 2 2000
Lonestar Amazed 4:25 2 54 1999-06-12 \#\textgreater{} 3 2000 Lonestar
Amazed 4:25 3 44 1999-06-19 \#\textgreater{} 4 2000 Lonestar Amazed 4:25
4 39 1999-06-26 \#\textgreater{} 5 2000 Lonestar Amazed 4:25 5 38
1999-07-03 \#\textgreater{} 6 2000 Lonestar Amazed 4:25 6 33 1999-07-10
\#\textgreater{} 7 2000 Lonestar Amazed 4:25 7 29 1999-07-17
\#\textgreater{} 8 2000 Amber Sexual 4:38 1 99 1999-07-17
\#\textgreater{} 9 2000 Lonestar Amazed 4:25 8 29 1999-07-24
\#\textgreater{} 10 2000 Amber Sexual 4:38 2 99 1999-07-24
\#\textgreater{} .. \ldots{} \ldots{} \ldots{} \ldots{} \ldots{}
\ldots{} \ldots{}

Multiple variables stored in one column

After gathering columns, the key column is sometimes a combination of
multiple underlying variable names. This happens in the tb
(tuberculosis) dataset, shown below. This dataset comes from the World
Health Organisation, and records the counts of confirmed tuberculosis
cases by country, year, and demographic group. The demographic groups
are broken down by sex (m, f) and age (0-14, 15-25, 25-34, 35-44, 45-54,
55-64, unknown).

tb \textless{}- tbl\_df(read.csv(``tb.csv'', stringsAsFactors = FALSE))
tb \#\textgreater{} Source: local data frame {[}5,769 x 22{]}
\#\textgreater{} \#\textgreater{} iso2 year m04 m514 m014 m1524 m2534
m3544 m4554 m5564 m65 mu f04 f514 \#\textgreater{} 1 AD 1989 NA NA NA NA
NA NA NA NA NA NA NA NA \#\textgreater{} 2 AD 1990 NA NA NA NA NA NA NA
NA NA NA NA NA \#\textgreater{} 3 AD 1991 NA NA NA NA NA NA NA NA NA NA
NA NA \#\textgreater{} 4 AD 1992 NA NA NA NA NA NA NA NA NA NA NA NA
\#\textgreater{} 5 AD 1993 NA NA NA NA NA NA NA NA NA NA NA NA
\#\textgreater{} 6 AD 1994 NA NA NA NA NA NA NA NA NA NA NA NA
\#\textgreater{} 7 AD 1996 NA NA 0 0 0 4 1 0 0 NA NA NA \#\textgreater{}
8 AD 1997 NA NA 0 0 1 2 2 1 6 NA NA NA \#\textgreater{} 9 AD 1998 NA NA
0 0 0 1 0 0 0 NA NA NA \#\textgreater{} 10 AD 1999 NA NA 0 0 0 1 1 0 0
NA NA NA \#\textgreater{} .. \ldots{} \ldots{} \ldots{} \ldots{}
\ldots{} \ldots{} \ldots{} \ldots{} \ldots{} \ldots{} \ldots{} ..
\ldots{} \ldots{} \#\textgreater{} Variables not shown: f014 (int),
f1524 (int), f2534 (int), f3544 (int), \#\textgreater{} f4554 (int),
f5564 (int), f65 (int), fu (int)

First we gather up the non-variable columns:

tb2 \textless{}- tb \%\textgreater{}\% gather(demo, n, -iso2, -year,
na.rm = TRUE) tb2 \#\textgreater{} Source: local data frame {[}35,750 x
4{]} \#\textgreater{} \#\textgreater{} iso2 year demo n \#\textgreater{}
1 AD 2005 m04 0 \#\textgreater{} 2 AD 2006 m04 0 \#\textgreater{} 3 AD
2008 m04 0 \#\textgreater{} 4 AE 2006 m04 0 \#\textgreater{} 5 AE 2007
m04 0 \#\textgreater{} 6 AE 2008 m04 0 \#\textgreater{} 7 AG 2007 m04 0
\#\textgreater{} 8 AL 2005 m04 0 \#\textgreater{} 9 AL 2006 m04 1
\#\textgreater{} 10 AL 2007 m04 0 \#\textgreater{} .. \ldots{} \ldots{}
\ldots{} .

Column headers in this format are often separated by a non-alphanumeric
character (e.g. ., -, \_, :), or have a fixed width format, like in this
dataset. separate() makes it easy to split a compound variables into
individual variables. You can either pass it a regular expression to
split on (the default is to split on non-alphanumeric columns), or a
vector of character positions. In this case we want to split after the
first character:

tb3 \textless{}- tb2 \%\textgreater{}\% separate(demo, c(``sex'',
``age''), 1) tb3 \#\textgreater{} Source: local data frame {[}35,750 x
5{]} \#\textgreater{} \#\textgreater{} iso2 year sex age n
\#\textgreater{} 1 AD 2005 m 04 0 \#\textgreater{} 2 AD 2006 m 04 0
\#\textgreater{} 3 AD 2008 m 04 0 \#\textgreater{} 4 AE 2006 m 04 0
\#\textgreater{} 5 AE 2007 m 04 0 \#\textgreater{} 6 AE 2008 m 04 0
\#\textgreater{} 7 AG 2007 m 04 0 \#\textgreater{} 8 AL 2005 m 04 0
\#\textgreater{} 9 AL 2006 m 04 1 \#\textgreater{} 10 AL 2007 m 04 0
\#\textgreater{} .. \ldots{} \ldots{} \ldots{} \ldots{} .

Storing the values in this form resolves a problem in the original data.
We want to compare rates, not counts, which means we need to know the
population. In the original format, there is no easy way to add a
population variable. It has to be stored in a separate table, which
makes it hard to correctly match populations to counts. In tidy form,
adding variables for population and rate is easy because they're just
additional columns. Variables are stored in both rows and columns

The most complicated form of messy data occurs when variables are stored
in both rows and columns. The code below loads daily weather data from
the Global Historical Climatology Network for one weather station
(MX17004) in Mexico for five months in 2010.

weather \textless{}- tbl\_df(read.csv(``weather.csv'', stringsAsFactors
= FALSE)) weather \#\textgreater{} Source: local data frame {[}22 x
35{]} \#\textgreater{} \#\textgreater{} id year month element d1 d2 d3
d4 d5 d6 d7 d8 d9 d10 d11 \#\textgreater{} 1 MX17004 2010 1 tmax NA NA
NA NA NA NA NA NA NA NA NA \#\textgreater{} 2 MX17004 2010 1 tmin NA NA
NA NA NA NA NA NA NA NA NA \#\textgreater{} 3 MX17004 2010 2 tmax NA
27.3 24.1 NA NA NA NA NA NA NA 29.7 \#\textgreater{} 4 MX17004 2010 2
tmin NA 14.4 14.4 NA NA NA NA NA NA NA 13.4 \#\textgreater{} 5 MX17004
2010 3 tmax NA NA NA NA 32.1 NA NA NA NA 34.5 NA \#\textgreater{} 6
MX17004 2010 3 tmin NA NA NA NA 14.2 NA NA NA NA 16.8 NA
\#\textgreater{} 7 MX17004 2010 4 tmax NA NA NA NA NA NA NA NA NA NA NA
\#\textgreater{} 8 MX17004 2010 4 tmin NA NA NA NA NA NA NA NA NA NA NA
\#\textgreater{} 9 MX17004 2010 5 tmax NA NA NA NA NA NA NA NA NA NA NA
\#\textgreater{} 10 MX17004 2010 5 tmin NA NA NA NA NA NA NA NA NA NA NA
\#\textgreater{} .. \ldots{} \ldots{} \ldots{} \ldots{} .. \ldots{}
\ldots{} .. \ldots{} .. .. .. .. \ldots{} \ldots{} \#\textgreater{}
Variables not shown: d12 (lgl), d13 (dbl), d14 (dbl), d15 (dbl), d16
\#\textgreater{} (dbl), d17 (dbl), d18 (lgl), d19 (lgl), d20 (lgl), d21
(lgl), d22 (lgl), \#\textgreater{} d23 (dbl), d24 (lgl), d25 (dbl), d26
(dbl), d27 (dbl), d28 (dbl), d29 \#\textgreater{} (dbl), d30 (dbl), d31
(dbl)

It has variables in individual columns (id, year, month), spread across
columns (day, d1-d31) and across rows (tmin, tmax) (minimum and maximum
temperature). Months with less than 31 days have structural missing
values for the last day(s) of the month.

To tidy this dataset we first gather the day columns:

weather2 \textless{}- weather \%\textgreater{}\% gather(day, value,
d1:d31, na.rm = TRUE) weather2 \#\textgreater{} Source: local data frame
{[}66 x 6{]} \#\textgreater{} \#\textgreater{} id year month element day
value \#\textgreater{} 1 MX17004 2010 12 tmax d1 29.9 \#\textgreater{} 2
MX17004 2010 12 tmin d1 13.8 \#\textgreater{} 3 MX17004 2010 2 tmax d2
27.3 \#\textgreater{} 4 MX17004 2010 2 tmin d2 14.4 \#\textgreater{} 5
MX17004 2010 11 tmax d2 31.3 \#\textgreater{} 6 MX17004 2010 11 tmin d2
16.3 \#\textgreater{} 7 MX17004 2010 2 tmax d3 24.1 \#\textgreater{} 8
MX17004 2010 2 tmin d3 14.4 \#\textgreater{} 9 MX17004 2010 7 tmax d3
28.6 \#\textgreater{} 10 MX17004 2010 7 tmin d3 17.5 \#\textgreater{} ..
\ldots{} \ldots{} \ldots{} \ldots{} \ldots{} \ldots{}

For presentation, I've dropped the missing values, making them implicit
rather than explicit. This is ok because we know how many days are in
each month and can easily reconstruct the explicit missing values.

We'll also do a little cleaning:

weather3 \textless{}- weather2 \%\textgreater{}\% mutate(day =
extract\_numeric(day)) \%\textgreater{}\% select(id, year, month, day,
element, value) \%\textgreater{}\% arrange(id, year, month, day)
weather3 \#\textgreater{} Source: local data frame {[}66 x 6{]}
\#\textgreater{} \#\textgreater{} id year month day element value
\#\textgreater{} 1 MX17004 2010 1 30 tmax 27.8 \#\textgreater{} 2
MX17004 2010 1 30 tmin 14.5 \#\textgreater{} 3 MX17004 2010 2 2 tmax
27.3 \#\textgreater{} 4 MX17004 2010 2 2 tmin 14.4 \#\textgreater{} 5
MX17004 2010 2 3 tmax 24.1 \#\textgreater{} 6 MX17004 2010 2 3 tmin 14.4
\#\textgreater{} 7 MX17004 2010 2 11 tmax 29.7 \#\textgreater{} 8
MX17004 2010 2 11 tmin 13.4 \#\textgreater{} 9 MX17004 2010 2 23 tmax
29.9 \#\textgreater{} 10 MX17004 2010 2 23 tmin 10.7 \#\textgreater{} ..
\ldots{} \ldots{} \ldots{} \ldots{} \ldots{} \ldots{}

This dataset is mostly tidy, but the element column is not a variable;
it stores the names of variables. (Not shown in this example are the
other meteorological variables prcp (precipitation) and snow
(snowfall)). Fixing this requires the spread operation. This performs
the inverse of gathering by spreading the element and value columns back
out into the columns:

weather3 \%\textgreater{}\% spread(element, value) \#\textgreater{}
Source: local data frame {[}33 x 6{]} \#\textgreater{} \#\textgreater{}
id year month day tmax tmin \#\textgreater{} 1 MX17004 2010 1 30 27.8
14.5 \#\textgreater{} 2 MX17004 2010 2 2 27.3 14.4 \#\textgreater{} 3
MX17004 2010 2 3 24.1 14.4 \#\textgreater{} 4 MX17004 2010 2 11 29.7
13.4 \#\textgreater{} 5 MX17004 2010 2 23 29.9 10.7 \#\textgreater{} 6
MX17004 2010 3 5 32.1 14.2 \#\textgreater{} 7 MX17004 2010 3 10 34.5
16.8 \#\textgreater{} 8 MX17004 2010 3 16 31.1 17.6 \#\textgreater{} 9
MX17004 2010 4 27 36.3 16.7 \#\textgreater{} 10 MX17004 2010 5 27 33.2
18.2 \#\textgreater{} .. \ldots{} \ldots{} \ldots{} \ldots{} \ldots{}
\ldots{}

This form is tidy: there's one variable in each column, and each row
represents one day. Multiple types in one table \{\#multiple-types\}

Datasets often involve values collected at multiple levels, on different
types of observational units. During tidying, each type of observational
unit should be stored in its own table. This is closely related to the
idea of database normalisation, where each fact is expressed in only one
place. It's important because otherwise inconsistencies can arise.

The billboard dataset actually contains observations on two types of
observational units: the song and its rank in each week. This manifests
itself through the duplication of facts about the song: artist, year and
time are repeated many times.

This dataset needs to be broken down into two pieces: a song dataset
which stores artist, song name and time, and a ranking dataset which
gives the rank of the song in each week. We first extract a song
dataset:

song \textless{}- billboard3 \%\textgreater{}\% select(artist, track,
year, time) \%\textgreater{}\% unique() \%\textgreater{}\%
mutate(song\_id = row\_number()) song \#\textgreater{} Source: local
data frame {[}317 x 5{]} \#\textgreater{} \#\textgreater{} artist track
year time song\_id \#\textgreater{} 1 2 Pac Baby Don't Cry (Keep\ldots{}
2000 4:22 1 \#\textgreater{} 2 2Ge+her The Hardest Part Of \ldots{} 2000
3:15 2 \#\textgreater{} 3 3 Doors Down Kryptonite 2000 3:53 3
\#\textgreater{} 4 3 Doors Down Loser 2000 4:24 4 \#\textgreater{} 5 504
Boyz Wobble Wobble 2000 3:35 5 \#\textgreater{} 6 98\^{}0 Give Me Just
One Nig\ldots{} 2000 3:24 6 \#\textgreater{} 7 A*Teens Dancing Queen
2000 3:44 7 \#\textgreater{} 8 Aaliyah I Don't Wanna 2000 4:15 8
\#\textgreater{} 9 Aaliyah Try Again 2000 4:03 9 \#\textgreater{} 10
Adams, Yolanda Open My Heart 2000 5:30 10 \#\textgreater{} .. \ldots{}
\ldots{} \ldots{} \ldots{} \ldots{}

Then use that to make a rank dataset by replacing repeated song facts
with a pointer to song details (a unique song id):

rank \textless{}- billboard3 \%\textgreater{}\% left\_join(song,
c(``artist'', ``track'', ``year'', ``time'')) \%\textgreater{}\%
select(song\_id, date, week, rank) \%\textgreater{}\% arrange(song\_id,
date) rank \#\textgreater{} Source: local data frame {[}5,307 x 4{]}
\#\textgreater{} \#\textgreater{} song\_id date week rank
\#\textgreater{} 1 1 2000-02-26 1 87 \#\textgreater{} 2 1 2000-03-04 2
82 \#\textgreater{} 3 1 2000-03-11 3 72 \#\textgreater{} 4 1 2000-03-18
4 77 \#\textgreater{} 5 1 2000-03-25 5 87 \#\textgreater{} 6 1
2000-04-01 6 94 \#\textgreater{} 7 1 2000-04-08 7 99 \#\textgreater{} 8
2 2000-09-02 1 91 \#\textgreater{} 9 2 2000-09-09 2 87 \#\textgreater{}
10 2 2000-09-16 3 92 \#\textgreater{} .. \ldots{} \ldots{} \ldots{}
\ldots{}

You could also imagine a week dataset which would record background
information about the week, maybe the total number of songs sold or
similar ``demographic'' information.

Normalisation is useful for tidying and eliminating inconsistencies.
However, there are few data analysis tools that work directly with
relational data, so analysis usually also requires denormalisation or
the merging the datasets back into one table. One type in multiple
tables

It's also common to find data values about a single type of
observational unit spread out over multiple tables or files. These
tables and files are often split up by another variable, so that each
represents a single year, person, or location. As long as the format for
individual records is consistent, this is an easy problem to fix:

\begin{verbatim}
Read the files into a list of tables.

For each table, add a new column that records the original file name (the file name is often the value of an important variable).

Combine all tables into a single table.
\end{verbatim}

Plyr makes this straightforward in R. The following code generates a
vector of file names in a directory (data/) which match a regular
expression (ends in .csv). Next we name each element of the vector with
the name of the file. We do this because will preserve the names in the
following step, ensuring that each row in the final data frame is
labeled with its source. Finally, ldply() loops over each path, reading
in the csv file and combining the results into a single data frame.

library(plyr) paths \textless{}- dir(``data'', pattern =
``\textbackslash{}.csv\$'', full.names = TRUE) names(paths) \textless{}-
basename(paths) ldply(paths, read.csv, stringsAsFactors = FALSE)

Once you have a single table, you can perform additional tidying as
needed. An example of this type of cleaning can be found at
\url{https://github.com/hadley/data-baby-names} which takes 129 yearly
baby name tables provided by the US Social Security Administration and
combines them into a single file.

A more complicated situation occurs when the dataset structure changes
over time. For example, the datasets may contain different variables,
the same variables with different names, different file formats, or
different conventions for missing values. This may require you to tidy
each file to individually (or, if you're lucky, in small groups) and
then combine them once tidied. An example of this type of tidying is
illustrated in \url{https://github.com/hadley/data-fuel-economy}, which
shows the tidying of epa fuel economy data for over 50,000 cars from
1978 to 2008. The raw data is available online, but each year is stored
in a separate file and there are four major formats with many minor
variations, making tidying this dataset a considerable challenge.

```

\subsection{Commentaires sur article de
Wicham}\label{commentaires-sur-article-de-wicham}

\href{http://serialmentor.com/blog/2014/7/20/keep-your-data-tidy}{commentaire
1}: petit exemple d'utilisation avec dipplyr, notamment summarize.

\href{http://www.dataschool.io/tidying-messy-data-in-r/}{autre
commentaire}

\href{http://stats.stackexchange.com/questions/83614/best-practices-for-creating-tidy-data}{stackoverflow}
Some points (from my experiences):

\begin{verbatim}
Some people like colorful spreadsheets and make abundant use of formatting options. This is all fine, if it helps them organize their data and prepare tables for presentation. However, it's dangerous if a cell color actually encodes data. It's easy to lose this data and very difficult to get such data imported into statistical software (e.g., see this question on Stack Overflow).
Sometimes I get some nicely formatted data (after I told people how to prepare it), but despite asking them to use a dedicated column or separate file for comments they decide to put a comment in a value column. Not only do I need to deal with this column in a special way when importing the data, but the main problem is that I would need to scroll through all the table to see such comments (which I usually wouldn't do). This gets even worse if they use Excel's commenting facilities.
Spreadsheets with several tables in them, multiple header lines or connected cells result in manual work to prepare them for import in statistical software. Good data analysts usually don't enjoy this kind of manual work.
Never, ever hide columns in Excel. If they are not needed, delete them. If they are needed, show them.
xls and its descendants are not suitable file formats for exchanging data with others or archiving it. Formulas get updated when the file is opened and different Excel versions might handle the files differently. I recommend a simple CSV file instead, since almost all data-related software can import that (even Excel) and it can be expected that that won't change soon. However, be aware that Excel rounds to visible digits when saving to a CSV (thereby discarding precision).
If you want to make life easy for others, adhere to the principles given in Hadley's article. Have a value column for each variable and factor columns defining strata
\end{verbatim}

traduction (ggole)

Quelques points (à partir de mes expériences):

\begin{verbatim}
Certaines personnes aiment tableurs colorés et font un usage abondant d'options de formatage. Tout cela est bien, si elle aide à organiser leurs données et de préparer des tableaux pour la présentation. Cependant, ce est dangereux si une couleur de cellule code réellement des données. Il est facile de perdre ces données et très difficile d'obtenir ces données importées dans un logiciel statistique (voir par exemple cette question sur Stack Overflow).
Parfois, je reçois des données correctement formatées (après que je ai dit aux gens comment préparer), mais malgré leur demandant d'utiliser une colonne dédiée ou fichier séparé pour les commentaires qu'ils décident de mettre un commentaire dans une colonne de valeur. Non seulement ai-je besoin pour faire face à cette colonne d'une manière particulière lors de l'importation des données, mais le principal problème est que je aurais besoin de faire défiler toute la table pour voir ces commentaires (que je avais l'habitude de ne pas faire). Ce est encore pire si elles utilisent les installations commentant d'Excel.
Feuilles de calcul avec plusieurs tables en eux, plusieurs lignes d'en-tête ou cellules connectées entraînent dans le travail manuel pour les préparer à l'importation dans le logiciel statistique. Données de bons analystes en général ne jouissent pas de ce genre de travail manuel.
Jamais, jamais masquer des colonnes dans Excel. Se ils ne sont pas nécessaires, supprimez-les. Si elles sont nécessaires, leur montrer.
xls et ses descendants ne sont pas des formats adaptés pour échanger des données avec d'autres ou de les archiver. Formules sont mises à jour lorsque le fichier est ouvert et différentes versions d'Excel peuvent gérer les fichiers différemment. Je recommande un fichier CSV simple, à la place, puisque presque tous les logiciels liés aux données peut importer que (même Excel) et on peut se attendre que cela ne changera pas bientôt. Toutefois, sachez que Excel arrondit les chiffres sont visibles lors de l'enregistrement d'un fichier CSV (rejetant ainsi la précision).
Si vous voulez rendre la vie facile pour les autres, respecter les principes énoncés dans l'article de Hadley. Avoir une colonne de valeur pour chaque colonnes variables et des facteurs définissant strates.
\end{verbatim}

\href{http://www.unomaha.edu/mahbubulmajumder/data-science/fall-2014/lectures/27-messy-data/27-messy-data.html\#/}{Handling
missing and messy data}

\section{Tidyr}\label{tidyr}

\href{http://blog.rstudio.org/2014/07/22/introducing-tidyr/}{nouveau
package}
\href{http://stackoverflow.com/search?tab=votes\&q={[}r{]}\%20tidyr}{tidyr}:
page spécifique sur stackoverflow

\href{https://github.com/jtleek/datasharing}{Jeffrey Leek} How to share
data with a statistician

\href{http://en.wikipedia.org/wiki/Data_cleansing}{Data cleansing}
article de wikipédia. Pas d'équivalent en français.

\href{http://dh-r.lincolnmullen.com/data.html}{Data manipulation}

\href{http://connor-johnson.com/2014/08/28/tidyr-and-pandas-gather-and-melt/}{tidyr
and pandas: Gather and Melt}

\section{Alternatives}\label{alternatives}

\href{http://www.ttdatavis.onthinktanks.org/how-tos/how-to-merge-and-tidy-data-with-excel}{How
to merge and tidy data with Excel} accédé le 1/3/2015.

\section{Divers}\label{divers-1}

\href{http://www.datakind.org/blog/whats-in-a-table/}{Anatomie d'une
table}

\end{document}
